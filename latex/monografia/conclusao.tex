Com os resultados descritos de forma técnica, uma conclusão sob estes é possibilitada. A proposta do trabalho, de verificar se treinar redes adversarias generativas especificamente para um determinado cenário produziria melhores resultados do que uma RAG treinada de forma genérica, foi fortemente desafiada pelos resultados, no quesito viabilidade. O trabalho documenta uma estratégia pouco otimizada de se realizar super resolução de imagens, especialmente se comparado ao estado da arte. No ponto de vista científico no entanto, este texto pode vir a conduzir futuros leitores para fora de um caminho pouco frutífero e que não trará resultados positivos. 

No que tangem os objetivos, sejam estes específicos ou geral, pode-se avaliar que o trabalho realizou a proposta apresentada. Alguns destes objetivos tiveram seus cumprimentos delongados devido a obstáculos impostos por diversos fatores técnicos. Outros, no entanto, foram cumpridos de maneira menos tortuosa. Este é o caso da definição das bases de imagens para treinamento dos modelos. Por mais que determinado esforço de pesquisa tenha sido necessário até que alternativas viáveis e cabíveis fossem encontradas e eventualmente as imagens nestas contidas fossem processadas para que se encaixassem nos requisitos impostos pelo modelo, todo esse processo foi antecipado nas fases de concepção deste trabalho.

Os objetivos específicos de definir o modelo a ser utilizado e de integração dos componentes de \textit{software} impuseram altos e baixos no fluxo de desenvolvimento. A definição do modelo envolveu muitas tentativas e erros, e vários candidatos em potencial foram descartados seja porque não funcionaram, ou porque se desempenharam de maneira pobre durante os testes preliminares, falhando seguidamente ou tomando muito tempo para falhar. Em um dos casos, o modelo treinou por horas durante a noite e falhou em algum momento da madrugada. Este objetivo anda de mãos atadas com o objetivo de integração, já que para que se defina um modelo que atenda ao escopo proposto, uma integração harmoniosa é indispensável. Como descrito detalhadamente na seção \ref{sec:environment-version-compatability}, um leque de compatibilidade teve de ser encontrado entre as variadas partes que compuseram o modelo utilizado e também os modelos descartados. Encontrar essa faixa de compatibilidade possui desafios próprios. Nem todo \textit{software} utilizado possui uma documentação clara e explícita sobre as bibliotecas que foram utilizadas em sua composição. Em consequência, mais uma vez, a tentativa e erro foi empregada de maneira exaustiva até que uma configuração funcional fosse estabelecida.

Na fase final do trabalho, o último objetivo que se refere à utilização de métodos de avaliação de similaridade de imagens para avaliação de custo-benefício se realizou. O processo em si envolveu bastante automação já que quantidades grandes de imagens estavam sendo testadas. Este objetivo é o último requisito para que se possa avaliar o objetivo geral do trabalho.  

Os resultados obtidos pelo TE (treinamento específico) foram constantemente inferiores aos resultados obtidos pelo TG (Treinamento genérico). Em todos os métodos de avaliação utilizados (MSE, RMSE, PSNR e ERGAS) o TE produziu um desempenho pior. Algumas hipóteses podem ser levantadas sobre as razões pelas quais tais resultados foram obtidos.

A primeira das hipóteses é a limitação de recursos computacionais para o treinamento específico. Para executar o treinamento na máquina disponível, apenas duas imagens eram colocadas em memória por iteração, ou seja, um \textit{batch size} de dois. No contrário, problemas de memória impediriam o treinamento. Para fins de comparação, o modelo treinado por \citeonline{wang_esrgan_2018}, utilizou um \textit{batch size} de 16 e o modelo de \citeonline{wang_real-esrgan_2021}, uma evolução do anterior, utiliza um \textit{batch size} de 48. 

Além desta limitação de memória, há também a limitação de processamento e, conectada a isso, as restrições de tempo. Redes neurais complexas como as que compõem a ESRGAN são, geralmente, treinadas em servidores de alto desempenho, contendo múltiplos chips gráficos com dezenas de milhares de núcleo. Os treinamentos realizados para este trabalho, foram feitas em uma máquina pessoal, ordens de grandeza menos potente que as máquinas utilizadas pela indústria. Enquanto treinamentos profissionais acontecem por dias, o dispositivo utilizado neste trabalho não estava disponível vinte e quatro horas por dia para o treinamento. Tudo isso acarretou na redução de mais um parâmetro de configuração: o número total de épocas.

Para ambas as bases de dados, o treinamento específico foi feito por 100 épocas com 1000 iterações cada, um número baixo em comparação à treinamentos mais profissionais. Apesar disso, o modelo pré-treinado, obrigatório para especializar as redes, pode ter ajudado o desempenho do TE não ter sido ainda mais inferior ao desempenho do TG. TG que por sua vez, foi feito por terceiros em um ambiente com recursos muito mais abundantes. Apesar de não ser conhecido o número de épocas exato utilizado no modelo treinado genericamente, é possível ter uma ideia aproximada deste valor. O artigo de \citeonline{wang_esrgan_2018} menciona treinamentos de 50000, 100000, 200000 e 300000 iterações. Um segundo artigo feito em cima de uma melhoria ao modelo ESRGAN \cite{wang_real-esrgan_2021} detalha treinamentos de 400000 e 1000000 de iterações. Valores totalmente impraticáveis para os recursos disponíveis para a execução deste trabalho. Um treinamento com esta duração envolveria um comprometimento temporal muito grande. Estes valores se traduziriam em semanas, quiçá meses de uso intenso e ``ininterrupto'' do dispositivo utilizado para o treinamento, algo que foge do escopo do presente trabalho. A palavra ``ininterrupto'' foi escrita entre aspas por um motivo. O treinamento de fato, não pode ser interrompido, porém, existem formas de configurar o projeto para salvar pontos de checagem do modelo. Isso abre as portas para um trabalho futuro: explorar treinamentos longos, porém pausados e analisar os resultados.

Ambos modelos treinados nos artigos citados anteriormente, contaram com recursos computacionais de alto nível para o treinamento das redes. Tais dispositivos reduzem consideravelmente o tempo de treinamento, como mencionado na seção \ref{sec:environment-prep-training}. 

Outra hipótese diz respeito à disparidade entre os erros da base de ressonância magnética e de astronomia. Comparando as seções \ref{sec:result:mri} e \ref{sec:result:astronomy}, especialmente os erros PSNR (seções \ref{sec:result:mri:psnr} e \ref{sec:result:astronomy:psnr}) nota-se que o TE sob a base astronômica, apesar de ter sido realizado no mesmo dispositivo, com os mesmos parâmetros, produziu resultados piores que o TE da base de ressonância magnética. Existem duas diferenças críticas entre as duas bases de dados. Primeiro, a base de dados astronômica possui imagens coloridas. Segundo, como visto na seção \ref{sec:desenvolvimento:general-considerations-post-processed-images}, as imagens de ressonância magnética são menores em disco que as imagens astronômicas. Ambas as diferenças impõem obstáculos para o TE das imagens astronômicas. 

Imagens coloridas, são mais complexas estruturalmente que imagens monocromáticas (vide diagrama da figura \ref{fig:monochrome_vs_color}). Considerando os mesmos recursos e parâmetros utilizados para um treinamento, é seguro dizer que o treinamento realizado sobre imagens coloridas produzirá resultados inferiores. Quando se trata de imagens coloridas, as dimensões das estruturas internas de dados aumentam em algumas vezes. Uma imagem de largura L, altura A em tons de cinza possui uma estrutura bidimensional L por A, cada posição contendo a intensidade do respectivo pixel. A mesma imagem, agora com três canais de cores (vermelho, verde e azul), possui a mesma estrutura, porém repetida três vezes, uma para cada canal. Logo, há mais informação para ser analisada e mais informação para as redes se especializarem. Um trabalho futuro pode ser extraído destas disparidades: refazer os treinamentos com bases de imagens mais uniformes.

O desempenho do TE, mesmo que inferior ao TG, poderia ter sido significativamente pior. Na seção de desenvolvimento \ref{sec:preping-project-for-training}, foi descrito que para treinar as redes, um modelo pré-treinado seria necessário, para que o treinamento em si não precisasse aprender do zero. Dessa forma, a redes puderam se especializar em cima de algo já treinado, mesmo que minimamente. 

O maior desafio deste trabalho não foi a revisão bibliográfica, não foi compreender conceitos complexos de aprendizado de máquina e também não foi o esforço para encontrar e processar bases de imagens acessíveis de forma pública e gratuita. Apesar de todas estas fases citadas terem requerido um esforço significativo, o maior obstáculo, e em consequência, onde o maior aprendizado foi concretizado, foram as integrações de softwares de terceiros um com o outro, podendo-se utilizar apenas da documentação provida pelos desenvolvedores. Documentação que embora essencial, não foi suficiente para garantir o funcionamento de ponta a ponta. Bastante tentativa e erro foi empregada neste processo de integração. 

O trabalho desenvolvido, aponta que dado recursos reduzidos ou disputados por outras tarefas, utilizar um modelo de super resolução baseado em redes geradoras adversárias do tipo ESRGAN treinado de forma genérica, trará resultados mais satisfatórios.


