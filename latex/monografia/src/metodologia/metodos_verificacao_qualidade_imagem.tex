\section{Métodos de verificação de qualidade de imagens}
\label{sec:qualidade-imagem}
\index{Similaridade de imagens}

Apesar da qualidade de uma imagem poder ser tratada como algo subjetivo do ponto de vista da fotografia como arte, existem formas de a avaliarmos quantitativamente. Neste trabalho há um interesse particular em avaliar a proximidade ou similaridade entre imagens. Considere as seguintes assunções:

\begin{enumerate}
    \item Seja \textbf{A} uma imagem de alta qualidade
    \item Seja \textbf{B} a versão comprimida e reduzida da imagem A
    \item Seja \textbf{C} o resultado da super resolução da imagem \textbf{B} por uma \textbf{ESRGAN}
\end{enumerate}

Tenha em mente, que as assunções acima se aplicam somente ao escopo deste trabalho. Os métodos estatísticos e quantitativos de calcular a similaridade entre imagens devem, em sua maioria calculá-la independentemente das imagens em questão, sejam versões comprimidas uma da outra ou sejam imagens completamente distintas e não relacionadas.

Para de verificar a eficiência e confiabilidade da super resolução obtida na imagem \textbf{C}, precisa-se de avaliar a similidade, ou disparidade entre a imagem \textbf{A}, original e descomprimida, e a imagem \textbf{C} super resolvida, após a compressão.

Existem vários métodos estatísticos que podem ser utilizados para se calcular a semelhança matemática entre duas imagens. Dessa forma, é possível medir numericamente o quão próximas são as imagens \textbf{A} e \textbf{C}.

Abaixo, alguns destes métodos estão descritos. Leve em consideração, que os nomes e siglas serão deixados como são mais conhecidos, e em consequência mais facilmente encontrados. Em sua maioria, os termos originais são em inglês, porém uma tradução está apresentada na descrição.

Existem diversas bibliotecas disponíveis para calcular estes índices de similaridade, como a biblioteca de \textit{Python}, \textit{sewar} \cite{khalel_sewar_2023}. Uma biblioteca de código aberto que implementa diversos índices de maneira simplória e minimalista.


\subsection{\textit{Mean Squared Error (MSE)}}
\label{alg:mse}
\index{MSE}
\index{Mean Squared Error}

O \textit{Mean Squared Error} (Erro quadrático médio, do inglês), pode ser definido da seguinte forma entre duas imagens (É bastante comum encontrar o termo \textit{sinal} no lugar de \textit{imagem}) \textbf{x} e \textbf{y}:

\begin{equation}
    MSE(x,y) = \frac{1}{n} \sum_{i=1}^{n} \left(y_{i}-y_{i}^{~}\right)^{2}
\end{equation}

No MSE, quanto menor o valor, menor o erro entre as duas imagens e em consequência, maior a similaridade.

\subsection{\textit{Root Mean Squared Error (RMSE)}}
\label{alg:rmse}
\index{RMSE}
\index{Root Mean Squared Error}

O \textit{Root Mean Squared Error} (Raiz do erro quadrático médio, do inglês) é uma variação do MSE(\ref{alg:mse}) e pode ser definido da seguinte forma entre duas imagens (Assim como o MSE(\ref{alg:mse}), é bastante comum encontrar o termo \textit{sinal} no lugar de imagem) \textbf{x} e \textbf{y}:

\begin{equation}
    RMSE(x,y) = \sqrt{\sum_{i=1}^{n} \frac{\left(y_{i}-y_{i}^{~}\right)^{2}}{n}}
\end{equation}

No RMSE, assim como no MSE(\ref{alg:mse}) quanto menor o valor, menor o erro entre as duas imagens e em consequência, maior a similaridade.

\subsection{\textit{Peak Signal-to-Noise Ratio (PSNR)}}
\label{alg:psnr}
\index{PSNR}
\index{Peak Signal-to-Noise Ratio}

O \textit{Peak Signal-to-Noise Ratio} \cite{wang_image_2004} (Relação sinal-ruído de pico, do inglês) pode ser definido da seguinte forma entre os sinais ou imagens \textbf{x} e \textbf{y}:

\begin{equation}
    PSNR(x,y) = 10\cdot \log_{10}\left(\frac{MAX_{x}^{2}}{MSE(x,y)}\right)
\end{equation}

No PSNR, quanto maior o valor, maior a similaridade entre as imagens.

\subsection{\textit{Erreur Relative Globale Adimensionnelle de Synthèse (ERGAS)}}
\label{alg:ergas}
\index{ERGAS}
\index{Erreur Relative Globale Adimensionnelle de Synthèse}

O \textit{Erreur Relative Globale Adimensionnelle de Synthèse} \cite{wald_quality_2000} (Erro adimensional de síntese global relativa, do francês) pode ser definido da seguinte forma entre os sinais ou imagens \textbf{x} e \textbf{y}:

\begin{equation}
\begin{split}
    ERGAS(x,y)      &= 100\cdot \frac{h}{l} \sqrt{\frac{1}{N} \sum_{i=1}^{N} \frac{RMSE(x,y)^{2}}{M(x,y)^{2}}}
\end{split}
\end{equation}

No ERGAS, quanto menor o valor -- mais próximo de zero -- maior a similaridade.