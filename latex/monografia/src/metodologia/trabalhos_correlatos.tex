\section{Trabalhos correlatos}

As principais fontes para este trabalho se concentram principalmente em livros e artigos recentes, assim como artigos de sites online em alguns pontos específicos. Serão utilizados também alguns materiais mais antigos, especialmente na parte onde estarei fundamentando o conhecimento base (neurônios artificiais, redes neurais etc.). Estes trabalhos e técnicas já estão estabelecidos no meio acadêmico e suas citações são necessárias. Os artigos científicos foram obtidos de três fontes principais: Google Acadêmico, Portal de Periódico CAPES e \textit{Arxiv}.

\citeonline{ledig_photo-realistic_2016} desenvolveram uma topologia de redes geradoras adversárias (seção \ref{sec:gan}) que possui desempenho superior à outras técnicas no âmbito de super-resolução de imagens. O trabalho publicado, está gratuitamente disponível no repositório \textit{Arxiv} e será de suma importância para este trabalho. Pode-se dizer que este artigo foi a principal inspiração para o tema. Posteriormente, em 2018, \citeonline{wang_esrgan_2018} resolveu otimizar o modelo gerador proposto por \citeonline{ledig_photo-realistic_2016}. O autor identificou que para a forma com a qual o artigo inicial calcula os erros da rede geradora, a arquitetura não está otimizada. Foi proposta então, uma otimização para o modelo, que traz melhor desempenho no treinamento e resultados mais satisfatórios.

\citeonline{ledig_photo-realistic_2016} propuseram um modelo denominado SRGAN, baseando-se em uma topologia moderna de redes neurais. Esta arquitetura, chamada em português de Redes Adversárias Geradoras ou RAGs (do inglês \textit{Generative Adversarial Networks}, ou GANs) por sua vez, foi idealizada por \citeonline{goodfellow_generative_2014} com o intuito de gerar dados com inteligência artificial e aprendizado de máquina. O trabalho de \citeonline{ledig_photo-realistic_2016} generaliza tal técnica para recuperar detalhes de imagens em baixa resolução com fidelidade nunca antes vista, nem quando o estado da arte foi colocado como parâmetro de comparação. O modelo em si faz uso de algumas arquiteturas de redes neurais como redes completamente convolucionais e redes convolucioanis completamente conectadas. 

\citeonline{wang_esrgan_2018} então, experimentando o trabalho de \citeonline{ledig_photo-realistic_2016} encontraram áreas de melhoria no modelo e propuseram uma atualização ao modelo SRGAN. Estas melhorias fazem com que o treinamento das redes seja mais eficiente e desempenhe melhor quando determinadas métricas são utilizadas para mensurar o quão eficaz o modelo está sendo. Este novo modelo foi chamado de ESRGAN. 

O trabalho de \citeonline{wang_esrgan_2018} é bem mais abrangente que apenas uma otimização de treinamento. Os autores descrevem problemas como alucinações no modelo SRGAN sobre o qual o estudo é baseado. Estas alucinações, deterioram o resultado final da imagem super resolvida e a distancam da imagem em alta resolução esperada.