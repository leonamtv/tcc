\section{Trabalhos correlatos}

As principais fontes para este trabalho se concentram principalmente em livros e artigos recentes, assim como artigos de sites online em alguns pontos específicos. Serão utilizados também alguns materiais mais antigos, especialmente na parte onde estarei fundamentando o conhecimento base (neurônios artificiais, redes neurais etc.). Estes trabalhos e técnicas já estão estabelecidos no meio acadêmico e suas citações são necessárias. Os artigos científicos foram obtidos de três fontes principais: Google Acadêmico, Portal de Periódico CAPES e \textit{Arxiv}.

Ledig, et al. \cite{ledig_photo-realistic_2017} desenvolveram uma arquitetura de redes geradoras adversárias (\ref{sec:gan}) que possui desempenho superior à outras técnicas no âmbito de super-resolução de imagens. O trabalho publicado, está gratuitamente disponível no repositório \textit{Arxiv} e será de suma importância para este trabalho. Pode-se dizer que este artigo foi a principal inspiração para o tema. Posteriormente, em 2018, \cite{wang_esrgan_2018} resolveu otimizar o modelo gerador proposto por \cite{ledig_photo-realistic_2016}. O autor identificou que para a forma com a qual o artigo inicial calcula os erros da rede geradora, a arquitetura não está otimizada. Foi proposta e implementada então, uma otimização para o modelo, que traz melhor desempenho no treinamento e resultados mais satisfatórios.

